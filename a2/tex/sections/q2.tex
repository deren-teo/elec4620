\section*{Question 2}
\fakesection{2}

Consider a linear phase FIR filter with an odd number of taps, $N=2k+1$, such that the order of the filter is even: $M=2k$. We are interested in the filter response at $z=1$ and $-1$. As seen in Question 1, linear phase implies symmetric or antisymmetric coefficients.

We first consider the symmetric case: $h[n]=h[M-n]$. When the coefficients are symmetric, the first $k-1$ coefficients are equal to the last $k-1$ coefficients in reverse order, leaving only the $k$-th coefficient unpaired. Hence,
\begin{align}
    H(z) = \sum_{n=0}^{2k} h[n] z^{-n}
         = \sum_{n=0}^{k-1} h[n](z^{-n} + z^{n-M}) + h[k] z^{-k}
\end{align}
By observing $H(1)$ and $H(-1)$, we find that neither $z=1$ nor $z=-1$ are zeros:
\begin{align*}
    H(1) &= \sum_{n=0}^{k-1} h[n](1 + 1) + h[k](1)
          = 2 \sum_{n=0}^{k-1} h[n] + h[k]
\intertext{and}
    H(-1) &= \sum_{n=0}^{k-1} h[n]\left((-1)^{-n} + (-1)^{n-M}\right) + h[k](-1)^{-k} \\
          &= 2 \sum_{n=0}^{k-1} h[n](-1)^{-n} + h[k](-1)^{-k}
\end{align*}
For $H(-1)$ above, note that $(-1)^{-n}$ and $(-1)^{n-M}$ are always equal because $M$ is even. Hence, there is no restriction on whether the filter may be high pass or low pass (or band pass).

Similarly, let us observe the antisymmetric case: $h[n]=-h[M-n]$. When the coefficients are antisymmetric, the first $k-1$ coefficients are equal to the negation of the last $k-1$ coefficients in reverse order. Furthermore, for anti-symmetry, the central coefficient $h[k]$ must be zero:
\begin{align}
    H(z) = \sum_{n=0}^{2k} h[n] z^{-n}
         = \sum_{n=0}^{k-1} h[n](z^{-n} - z^{n-M}) + \cancelto{0}{h[k]} \cdot z^{-k}
\end{align}
This time, observing $H(1)$ and $H(-1)$, we find that both $z=1$ and $z=-1$ are zeros:
\begin{align*}
    H(1) &= \sum_{n=0}^{k-1} h[n](1 - 1)
          = 0
\intertext{and}
    H(-1) &= \sum_{n=0}^{k-1} h[n]((-1)^{-n} + (-1)^{n-M})
           = 0
\end{align*}
This cannot be a high pass or low pass filter, because both the low and high frequencies are attenuated by their proximity to the zeros at $z=1$ and -1.

\textbf{Thus, we reach our first conclusion; for a high or low pass linear phase FIR filter:}
\begin{enumerate}[label=\textbf{\roman*)}]
    \item \textbf{A filter with an odd number of taps has no zeros at $z=1$ or $-1$.}
\end{enumerate}

\newpage

Now consider a linear phase FIR filter with an even number of taps, $N=2k$, such that the order of the filter is odd: $M=2k-1$. Again, we are interested in the filter response at $z=1$ and $-1$.

We first consider the symmetric case: $h[n]=h[M-n]$. When the coefficients are symmetric, the first $k$ coefficients are equal to the second $k$ coefficients in reverse order. Hence,
\begin{align}
    H(z) = \sum_{n=0}^{2k-1} h[n] z^{-n}
         = \sum_{n=0}^{k-1} h[n] (z^{-n} + z^{n-M})
\end{align}
By observing $H(1)$ and $H(-1)$, we find that $z=-1$ is a zero but $z=1$ is not:
\begin{align*}
    H(1) &= \sum_{n=0}^{k-1} h[n](1 + 1) = 2 \sum_{n=0}^{k-1} h[n]
\intertext{and}
    H(-1) &= \sum_{n=0}^{k-1} h[n]((-1)^{-n} + (-1)^{n-M}) = 0
\end{align*}
For $H(-1)$ above, $(-1)^{-n}$ and $(-1)^{n-M}$ always have opposite sign because $M$ is odd. This implies a linear phase FIR filter with symmetric coefficients and even length must be low pass (or at least cannot be high pass), because the high frequencies near $z=-1$, corresponding to the Nyquist frequency, are attenuated by their proximity to a zero.

Similarly, let us observe the antisymmetric case: $h[n]=-h[M-n]$. When the coefficients are symmetric, the first $k$ coefficients are equal to the negation of the second $k$ coefficients in reverse order:
\begin{align}
    H(z) = \sum_{n=0}^{2k-1} h[n] z^{-n}
         = \sum_{n=0}^{k-1} h[n](z^{-n} - z^{n-M})
\end{align}
Once more, observing $H(1)$ and $H(-1)$, we find that $z=1$ is a zero but $z=-1$ is not:
\begin{align*}
    H(1) &= \sum_{n=0}^{k-1} h[n](1 - 1) = 0
\intertext{and}
    H(-1) &= \sum_{n=0}^{k-1} h[n]((-1)^{-n} - (-1)^{n-M}) \\
          &= 2 \sum_{n-0}^{k-1} h[n](-1)^{-n}
\end{align*}
This implies a linear phase FIR filter with antisymmetric coefficients and even length must be high pass (or at least cannot be low pass), because the low frequencies near $z=1$, corresponding to DC, are attenuated by their proximity to a zero.

\textbf{Thus, we reach our second and third conclusions; for a linear phase FIR filter:}
\begin{enumerate}[label=\textbf{\roman*)}] \setcounter{enumii}{1}
    \item \textbf{A high pass filter with an even number of taps has a zero at $z=1$.}
    \item \textbf{A low pass filter with an even number of taps has a zero at $z=-1$.}
\end{enumerate}
